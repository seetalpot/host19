\section{Introduction}
\label{sec:introduction}
\noindent 
%The fabless fabrication stragies were originally invented by Xilinx co-founder Bernie Vonderschmitt in 1993. 
%Since then, many companies have adopted them to improve their economies of scale, amidst the growing economic demands of maintaining independent foundry in a scaled technology. 
Today, fabless model is the most economical and hence the preferred business model of the semiconductor industry. 
However, IC counterfeiting, piracy and overbuilding in the untrusted offshore foundry have caused major concerns in electronic and defense industries~\cite{pecht06, trimberger07, farinaz:epic}. 

Logic encryption uses a low-overhead combinational chip-locking system, to combat these issues~\cite{farinaz:epic}. 
Several logic encryption have been proposed so far in literature~\cite{farinaz:epic, jv:dac12, jv:tc15, dupuis:iolts14, baumgarten:dandt10}, and each of them have their own merits and demerits. 
The SAT-attack~\cite{pramod:host15} is shown to successfully decrypt the logic encryption keys in all above cases, with over $95 \%$ success rate. 
Recently, sequential locking~\cite{rajit:encryptFF} was proposed as a defense to SAT-attack on sequential circuits. 
The flip-flop encryption is done in this paper, by inserting an XOR gate right at the output of selected flip-flops. 

A major limitation of this technique is that $58-100\%$ of flip-flops were encrypted, which 
induces huge are overhead in real designs. Apart from that, the most recently proposed ScanSAT method~\cite{lilas:aspdac19} 
converts the sequential locking problem to an instance of logic locking problem, and thereby decrypts a {\em functionally correct key} using SAT-attack. 
Thus, it is imperative to come up with a new sequential locking scheme that is area-efficient, as well as, and more importantly resilient to the SAT-attack. 

In this paper, we propose a sequential locking scheme that addresses the aforementioned challenges. The main contributions of this paper are as follows:
\begin{enumerate}
\item We show that ScanSAT does not guarantee decryption of {\em functionally correct key}, in the presence of encrypted XOR-chains at combinational outputs; 
\item We show there is $100\%$ correlation between failure of ScanSAT and presence of encrypted XOR-chains at combinational outputs; 
\item Exploiting this property as a defense mechanism, we propose a new low-overhead encrypted scan cell, that consists of an unencrypted functional path and an encrypted scan path; 
\item Finally, to increase the success rate of successful functional corruption, we propose an iterative combinational key gate pushing algorithm, that makes the sequential circuit resilient to SAT-attack. 
\end{enumerate}

