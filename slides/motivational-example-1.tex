\begin{frame}{Motivational example on SK routing}
\begin{itemize}
\item \texttt{ibm04} benchmark, die size including pins is 2612x2340; excluding pins is 2602x2321
\item a27189 cell is located at (1810,1121) after detailed placement of ibm04
\item \texttt{ibm04-encrypt} : a27189 cell encrypted using p288 terminal
\item After detailed placement of this encrypted circuit - die size including pins is 2612x2340 (same); excluding pins is 2603x2321
\item So, there is little/no impact for encrypting just one cell. Reason: there is already some spare routing space available
\end{itemize}
\end{frame}

\begin{frame}{Brainstorming}
\begin{itemize}
	\item \alert{Combinational logic encryption - check if somebody has done analysis on routing overhead}
	\begin{enumerate}
		\item How does routing overhead increase with increase in key length ?
		\item Tradeoff between security and routing overhead ?
		\item \texttt{IBM} benchmarks and {\em mPL} can be used for evaluation, setup is ready 
	\end{enumerate}
\item DFT encryption 
	\begin{enumerate}
		\item How does routing overhead increase with increase in key length ?
		\item Tradeoff between security and routing overhead ?
	\end{enumerate}
\end{itemize}
\end{frame}

