\begin{frame}[allowframebreaks]
\frametitle<presentation>{References}

\begin{thebibliography}{1}

\bibitem{halter:97}
J.P. Halter, F.N. Najm, ``A gate-level leakage power reduction method for ultra-low-power CMOS circuits'',  Custom Integrated Circuits Conference, IEEE, 1997, pp.475-478.

\bibitem{yibin:98}
Y. Ye, S. Borkar, and V. De, ``A new technique for standby leakage reduction in high-performance circuits,'' Symposium on VLSI Circuits, IEEE, 1998, pp. 40-41.

\bibitem{tschanz:03}
J.W. Tschanz, S.G. Narendra, Y. Ye, B.A. Bloechel, S. Borkar, and V. De, ``Dynamic sleep transistor and body bias for active leakage power control of microprocessors,'' IEEE Journal of Solid-State Circuits, vol. 38, Nov. 2003, pp. 1838-1845.

\bibitem{usami:06}
K. Usami and N. Ohkubo, ``A Design Approach for Fine-grained Run-Time Power Gating using Locally Extracted Sleep Signals,'' International Conference on Computer Design, IEEE, 2006, pp. 155-161.

\bibitem{bhunia:05}
S. Bhunia, N. Banerjee, Q. Chen, H. Mahmoodi, and K. Roy, ``A novel synthesis approach for active leakage power reduction using dynamic supply gating,'' Design Automation Conference, IEEE, 2005, pp. 479-484.

\bibitem{leinweber:08}
L. Leinweber and S. Bhunia, ``Fine-Grained Supply Gating Through Hypergraph Partitioning and Shannon Decomposition for Active Power Reduction,'' Design, Automation and Test in Europe, IEEE, 2008, pp. 373-378.

\bibitem{borkar:06}
S. Borkar, ``Tackling variability and reliability challenges'', IEEE Design and Test of Computers, vol.23, no.6, 2006, pp.520

\bibitem{rohfleisch:96}
B. Rohfleisch, A. Kolbl, B. Wurth, ``Reducing power dissipation after technology mapping by structural transformations'', Design Automation Conference Proceedings , IEEE, 1996, pp.789-794.

\end{thebibliography}
\end{frame}

